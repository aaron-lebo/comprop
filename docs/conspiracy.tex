\section{Introduction}

In December 2019, an unidentified virus was detected in China.
The initial cases were in Wuhan, a metro area with more than 11 million people, home to the Hunan Seafood Wholesale Market and the Wuhan Institute of Virology. Both were claimed to be the origin of the virus, analyses suggest it jumped from bats to humans as early as November \cite{parker2022}.
The Chinese medical community recognized it as novel in December, the first public notices of a ``pneumonia'' on the 31st of that month brought the virus to the attention of Chinese news outlets and social media before it was reported on internationally \cite{sahin2020}.

In January 2020 the virus was designated ``2019-nCoV'' \cite{world2020}.
Especially dangerous to the elderly and those with comorbidities, the first reported death was a 61-year old customer of the seafood market on January 9 \cite{burki2020}.
Four days later, Thailand reported the first case outside of China, a resident of Wuhan that had traveled to Bangkok by plane \cite{hsu2020}.
By the 20th, the China National Health Commission had confirmed the coronavirus was human-to-human transmissible and the United States confirmed its first case \cite{tang2020} \cite{stokes2020}.
It was declared a public health emergency in the US on January 31, and The World Heath Organization declared it a pandemic on March 11, 2020 \cite{patel2020} \cite{whaibeh2020}.
Despite travel bans, lockdowns, safety procedures including enforced masking, and the development of an innovative mRNA vaccine, efforts have failed to prevent the spread of multiple waves of variants.
Statistics have been inexact for a variety of reasons, but as of March 2023 the ongoing pandemic had caused a reported 6.88 million deaths globally and more than 1.12 million deaths in the United States \cite{dong2020}.

The initial reports on social media were shocking, with stories of people dropping in the streets and being sealed into their homes.
In the context of an ongoing trade war between the US and China, the Trump administration's scandals, the economic hardships brought by the pandemic, and the challenges of global collaboration, Covid-19 was heavily politicized.
Unclear and inconsistent messaging by authorities not only reduced social trust, but also fueled a growing public acceptance of conspiracy theories.

\subsection{What is a Conspiracy (Theory)?}

Conspiracy theorists say conspiracies happen all the time, and in historical terms, they do.
Conspiracy theories are sometimes portrayed as particularly American but they are not.
Conspiracies and conspiracy theories are often conflated, the term ``conspiracy'' and qualifiers ``theory'' and ``theorist'' can be used to delegitimize, discredit, and deflect valid criticism \cite{coady2006}.
Some self-identify as ``conspiracy theorists'' while others try to distance themselves from the label \cite{douglas2019}. 
Therefore, it is crucial to clearly define and delineate the two.

Conspiracies are secret plans by two or more actors to influence events \cite[p. 5]{pidgen1995} \cite[p. 116]{keeley1999}.
The group of conspirators is powerful enough for the conspiracy to be a threat.
While conspiracies can technically happen at any scale, in this context a conspiracy is not a plot to rob the corner store or Santa Claus, but rather is an attempt to ``usurp political or economic power, violate established rights, hoard vital secrets, or unlawfully alter government institutions'' \cite[p. 4]{douglas2019} \cite[p. 206]{sunstein2009} \cite[p. 31]{uscinski2014}.
Even those conspiracies that fail to be executed are considered to have occurred \cite{levy2007}.
Though they may have positive results, in democracies they are presumed suspect \cite{pidgen1995}.

History provides us with an always growing set of examples.
Elizabeth I of England's reign was a ``catalog of conspiracies'' \cite{pidgen1995}.
Indeed, every coup, counter-coup, revolution, and many assassinations are the product of at least one conspiracy. 
Countries conspire against countries as Nazi Germany and the Soviet Union did dividing up Poland in 1939, they were not the first to do so.
The French diplomat Talleyrand betrayed every regime he served from the 1790s to 1830s, usually for the benefit of France, but always for his own.
Theodore Roosevelt's administration conspired with revolutionaries in Panama who declared independence from Colombia, giving the United States the preferred route to build a canal over an alternative in Nicaragua.
Caesar's death on the Ides of March is both history and mythology having been retold so many times.
In Dante's \emph{Inferno}, his placement of Brutus and a fellow conspirator with Judas in the final circle of Hell shows the low historical reputation of those whom would conspire \cite{dante}.

The term conspiracy theory has scientific basis in Popper's (1972) ``conspiracy theory of society'' in which an all-powerful force produces history by conspiracy \cite{popper2014}.
Conspiracy theories propose to explain social phenomena as results of those plots, though it may fit some ideas like the Illuminati, his definition is overly restrictive \cite[pp. 5-7]{pidgen1995}.
More broadly, they are the explanation of ultimate causes of important events through conspiracies \cite[p. 116]{keeley1999}.
These theories are more like accusations, suggestions, or presumptions of supposed conspiracy, Goertzel calls them conspiracy beliefs \cite[p. 731]{goertzel1994}.
``Conspiracy refers to a true causal chain of events, a conspiracy theory refers to an allegation of conspiracy that may or may not be true'' \cite[p. 4]{douglas2019}.

The assassination of John F. Kennedy on November 22, 1963 has been the center of conspiracy belief for decades.
The 1964 Warren Commission concluded that Oswald acted alone while the 1979 United States House Select Committee on Assassinations concluded that Kennedy was probably killed in a conspiracy (though not by an organization connected to governments or crime) \cite{warren1964} \cite{select1979}.
Oliver Stone's blockbuster, \emph{JFK}, a conspiracy-focused retelling of the event reflected a public interest in and distrust of the official account which only grew over time. 
In 1966, a Gallup poll found 36\% believed Oswald acted alone, a number which fell to 10\% in a 1992 New York Times national survey despite the slow build-up of evidence backing the lone assassin narrative \cite[pp. 731-732]{goertzel1994}.

The events of Kennedy's death are a perfect storm for conspiracy theories.
There were many parties that had motive: Johnson succeeded him, Castro and his regime remembered the Bay of Pigs, the USSR was a constant threat, Robert Kennedy had gone after the Italian mafia as Attorney General, and ultra conservative groups disliked the president for reasons not limited to his Catholicism.
Oswald was an ex-marine that defected to the Soviet Union, and was visible enough in this and his work for the pro-Castro Fair Play for Cuba Committee that he was under surveillance by the FBI and CIA, these agencies would conduct the investigation for the Warren Commission.
He denied being the assassin, claimed to be a ``patsy'' after his arrest, and was killed two days later by Jack Ruby, a nightclub owner with unclear ties to local mafia \cite{gagne2022}.

This group of actors was by rumors, claims, and associations then connected to other groups, events, and people to spawn an untold number of theories. 
Johnson was said to have made strange statements in the aftermath, Kennedy's secretary said that the president had informed her that Johnson would not be his running mate in 1964 \cite{gagne2022}.
George H. W. Bush, later Director of the CIA, was supposedly in Dallas the day before and nearby Tyler the day of the assassination \cite[p. 54]{baker2010}.
Former CIA agent and Watergate burglar E. Howard Hunt's recorded deathbed confession in 2007 claimed knowledge of a conspiracy, and over the decades multiple people involved in organized crime have claimed mafia involvement. 
Finally, there were a variety of claims about the event: the supposed figures on the Grassy Knoll, accounts of multiple shooters, sightings of Oswald at other places, and conspicuous figures like the ``Umbrella Man''.
The latter would come forward in 1979 to reveal himself innocent \cite{gagne2022}.
Charles Manson's prosecutor estimates 42 various groups and 82 assassins have been alleged at different times \cite{patterson2013}.
One of these given narratives might be accurate, but despite their apparent research and vetting, dozens are contradictory and wrong.

\subsection{Conspiracism and the Paranoid Style}

The tendency for the adoption of one conspiracy leading to the clustering of beliefs has been observed repeatedly, the strongest predictor for the adoption of one conspiracy is the prior belief in another \cite[p. 953]{oliver2014}.
This pattern has been called conspiracy thinking, a conspiracy mindset, conspiracist ideation, the growing social acceptance: conspiracism, and has crossover with Popper's conspiracy theory of society \cite[p. 5]{douglas2019} \cite[p. 953]{oliver2014}.
Certain pathologies and personalities including paranoid personality disorder, schizotypy, nonclinical delusional thinking, narcissism, and social narcissism are vulnerable to conspiracy ideation \cite{zonis1994} \cite{barron2014} \cite[p. 8]{douglas2019}.
However, this psychological basis is not preferred, it fails to explain widespread social acceptance by otherwise normal individuals and groups.  

Historian Richard Hofstadter's \emph{The Paranoid Style in American Politics}, first released as an essay in Harper's Weekly in 1964 is one of the most influential works on conspiracism.
The paranoid style relates to the exaggeration, suspicion, fantasy, and feeling of persecution of the advocate, rather than the the content or truth of their message. 
While the style is found internationally, the apparent tendency in American politics to it has roots in smaller political groups. 
In 1798, at the height of partisanship and the threat of war, a hysteria against the Illuminati, Jeffersonian democracy, and French Jacobism included a 4th of July speech by the president of Yale.
The anti-Masonry of the 1820s and 1830s found an enemy in Jackson and old infamous Masons like Burr, though unlike the Illuminati fear, was not restricted to New England and was populist. 
Similarly, the populists at the end of the 19th century focused their ire on Eastern banking interests, but they had many other targets.
Protestant fear of Catholicism coexisted with the others and would regularly be inflamed well into the 20th century \cite{hofstadter2012}.

Over time mass media made enemies more concrete and right-wing conspiracies shifted from protecting an existing order to the talk of loss, dispossession. 
Socialists and communists had provoked scares prior to the New Deal, but peaked with Joseph McCarthy in the early 1950s.
McCarthy imagined a government overrun with communist agents and a military dangerously weak due to reduction from its wartime zenith.
He would be disgraced within a few years, but his claim of former Secretary of State George Marshall being a communist operative would inspire others, not even Eisenhower escaped accusation \cite{hofstadter2012}.

Hofstadter has been criticized for oversimplifying his analysis.
In fact, one of the groups he uses as an example, the Populists, were not wrong that the free silver movement was being manipulated, but rather about who was doing it, their ally \cite[pp. 3-4]{decanio2011}.
This critique ignores that the paranoid style is explicitly about delivery and messaging, not content, and the many populist demagogues of the era, including Ben Tillman, Tom Watson, James Hogg, James Vardaman, and their disciples are remembered for their many hatreds and paranoias \cite[pp. 44-45]{luthin1951}.
Regardless, this style can have productive, unintended effects. 
The rise of the free silver movement after the Civil War removed Democratic opposition to government intervention, William Jennings Bryan used this new freedom in 1896, 1900, and 1908 to call for government aid to the poor, sparking the progressive movement \cite[pp. 22-23]{decanio2011}.
Demonstrating this connection and the complexities of history, Woodrow Wilson's victory over Taft and Roosevelt in 1912 brought progressivism to power and ensured Republican conservatism, but Wilson was racist in belief and policy.

Historically, the American distrust of political authorities led to conspiracism across demographics and political attitudes \cite[p. 952]{oliver2014} \cite{goertzel1994}.
55\% of Americans polled in 2011 were believers in a conspiracy theory \cite[p. 956]{oliver2014}.
The United States is not unique, theories are common across the world \cite[p. 824]{miller2016} \cite{byford2001} \cite{zonis1994}.
Muslim countries seem especially vulnerable: 78\% of respondents to a poll in seven Muslim countries did not believe Arabs were responsible for 9/11, there is a recognizable literature on conspiracy theories in the Middle East \cite{gentzkow2004} \cite{zonis1994} \cite{pipes1996}.

\subsection{Why Believe in Conspiracy Theories?}

Of course there are the ``conspiracy entrepreneurs'' cited throughout that have the most to gain, true believers or not \cite[p. 213]{sunstein2009}.
These include actors as small as the vendor of JFK memorabilia in Dealey Plaza, but scholars have begun to view them in terms of rational social strategies.
Conspiracism is perhaps just another form of public discourse which gives framing to events \cite{erikson2015}.
It also serves as a distinctive mode of foreign policy by disinformation as Russia has demonstrated \cite{sakwa2012} \cite{watanabe2018}.
If we think of conspiracy theories in terms of a ``populist theory of power'', ``the people'' versus ``the Other'', they can expose inequities in the system and reallocate power to social and economic ``losers'' \cite{fenster2008} \cite{smallpage2017}.

Conspiracy theories, consciously or not, seem to fit needs that are epistemic (of understanding and certainty), existential (of control and safety), and social (of self and group image) in nature \cite[p. 7]{douglas2019} \cite{swami2010}.
Conspiracy belief increases with uncertainty and is associated with anomie, or a feeling of lacking a place in society and politics \cite[p. 8]{douglas2019} \cite[p. 739]{goertzel1994}.
It is appealing because not only do theorists have ``runs on the board'', but the theories themselves are self-sealing \cite[p. 132]{clarke2002} \cite[p. 7]{douglas2019}.
Flaws in official narratives are pointed out while the challenging theory offers a self-correcting and unifying explanation \cite[pp. 119, 135]{keeley1999}.
Theorists often note success in singular predictions being correct while (as in the JFK example) the majority of claims are false.

With the exception of ideological beliefs like Birtherism and Trutherism, believers in conspiracies are no less informed about political facts \cite[pp. 219, 264]{oliver2014}. 
Occasionally conspiracy theories can be productive in a scientific sense, but generally they are ways of finding accuracy and meaning in environments where more rational means are limited \cite{abalakina1999} \cite[p. 740]{goertzel1994}
Given the history of conspiracies by the US government (Operation Northwood, MKULTRA), especially towards minority groups (the Tuskegee Syphilis Experiment), it is not irrational for disadvantaged groups to have similar fears \cite{thomas1991}.
Individuals are more likely to believe conspiracies about their group if they have personally been discriminated against and conspiracies about other groups if their group has been treated unfairly \cite{simmons2005}.
There was widespread belief in the early 1990s in the gay and black communities that AIDS was caused by the government \cite[p. 1499]{thomas1991} \cite{nattrass2012}.
This was from a specific Soviet disinformation campaign, but minority status has been more generally associated with adoption of a higher number of conspiracies, relating to anomie, trust level, and employment concerns \cite[p. 739]{goertzel1994}.

\subsection{Flawed Thinking}

Despite conspiracism being an effective strategy at times, it has fundamental flaws.
It thrives on errors like confirmation bias, the conjunction fallacy, and motivated reasoning, the latter an effect most seen in partisan contexts in the US \cite[p. 826]{miller2016}.
Belief can also be a form of projection: the believer would conspire, therefore they expect others to do the same \cite[p. 7]{douglas2019}.
More problematically, the pattern of a state of crisis leading to an ``Other'' to serve as an enemy leads to dehumanization \cite[pp. 247-249]{graumann1987}.

Our in-group, out-group, othering dynamic was a protective necessity in an era before norms, human rights, and international law.
However, at scale, it runs into the fundamental attribution error: humans systematically believe dispositional factors (``this person would act this way'') over situational factors \cite[p. 143]{clarke2002}.
Especially in unfamiliar contexts, we explain events in terms of the supposed psychology of the actors doing them \cite{nisbett1991}.
This not only deprives one-off events of their unique contexts, but the unifying nature of conspiracy theories make the dispositional explain more than those individual events \cite[p. 146]{clarke2002}.

Furthermore, we make systematic errors taking small samples we are exposed to as representative of larger populations \cite{kahneman1982}.
Overcoming this and the fundamental attribution error goes against our cognitive instincts.
These instincts may have been formed in tribal groups where we interacted with a small number of people.
By presupposing someone's hostile predisposition as accurate, you were at worst overly cautious both within your family group and when dealing with outsiders \cite[pp. 146-147]{clarke2002}.

The othering brings us back to Popper's original criticism of conspiracy theories: they underestimate unintended consequences and assume intent as the cause \cite{popper2014}.
We tend to look for intentional agency, to associate individuals with ideas, and personify events, seeking a responsible party.
We wonder who benefits (``Cui bono?'') from each action \cite{graumann2012}.
However, in the modern world's massive, uncountable set of interacting agents, including political, social, and fact-gathering institutions, the notion of any single party controlling all is unrealistic, more so should they need to be secretive \cite{clarke2002}.
Such a large network will naturally lead to random interactions which appear planned and concerted to human pattern matching, made worse because its scale and our awareness of it is only a recent possibility \cite{taleb2008}.

In terms of being a productive way of thinking, the unified story that conspiracy theories provide should give pause.
Because humans are fallible, because we cannot account for all available data, theories that claim to explain all are wrong.
Conspiracy thinking often gets around this by claiming that any flaws are result of the Other, actively manipulating the data.
This is in contrast to scientific exploration where Nature is an uninterested bystander and makes conspiracy theories non-falsifiable \cite[pp. 120-121]{keeley1999}.
Lakatos refers to these as ``degenerating research programmes'', instead of updating Kuhnian scientific paradigms with new predictions to protect from challenges, the supporting data are interpreted to fit the theory \cite{lakatos1976}.

Conspiracy theories offer the same ``hackneyed explanation'' for every issue, whether it be the Jews, the mainstream media, or the globalists \cite[p. 741]{goertzel1994}.
This form of over rationalization is irrational, it is a form of ``global philosophical skepticism'' that is both arrogant and naive.
The conspiracy theorist pretends to be aware of manipulation on a massive scale, yet is hyper skeptical of conflicting information because of that.
The theorist pretends to know the psychology and intentions of various actors and weighs in on multiple high-level topics with absolute confidence, but their defining characteristic is usually pedantry and style of presentation.
Such an approach is attractive when juxtaposed against a seemingly random, meaningless world where a malcontent like Oswald can kill the leader of a superpower \cite[p. 120]{keeley1999}.

\subsection{The Terrorism Connection and the Rise of Q}

Conspiracy theories, propaganda, and terrorism often feed off of each other, especially in the era of social media, and it is the topic of terrorism that we focus on now.
Scholars have used many ways of modeling terrorism, but one of the most intuitive is the temporal system of Rapoport (2001), who divides terrorism into roughly four waves.
While terrorism has spanned the length of human existence, his first wave begins in the 1890s with anarchist movements in Russia (covered in Chapter 2) which spread throughout the Western world and lasted until the 1920s.
This wave was a struggle and statement against repressive regimes and the societies that supported them.
While this wave does not have a great hold on modern imaginations, two of its acts had major consequences: the assassination of William McKinley in 1901 led to the rise of Theodore Roosevelt and the modern American presidency, and the assassination of Franz Ferdinand led to World War I and our modern international system.
The second wave lasted from 1920 to 1960 and encompassed uprisings against colonial and foreign powers, such as the native Algerian struggle against the French.
The defeat of the United States in Vietnam set off a third wave lasting from the 1960s to the 1980s which saw a renewal of the revolutionary fervor of the first with heavy influence by Marxism.
Finally, the fourth wave started in the 1980s and lives into today.
It is dominated by Islamic fundamentalism and perhaps its most well-known features are spectacular attacks with massive body counts recently demonstrated with the rise and fall of ISIS \cite{rapoport2001}.

The wave system has also been used in the context of location.
The first anarchist wave started on the periphery of world power in Russia and spread to the center of power in the West.
The second wave of anti-colonial resistance mostly resided in the 
periphery, while the third wave of leftist terrorism was centered in Europe. Today’s wave of religious terrorism echoes the first in that initial incidents were in the periphery but has since shifted to or targets the West \cite{lizardo2003}.

Rapoport's orignal model is now 20 years old.
During that time, the United States has seen the rise of the domestic terrorist.
This actor is usually inspired by religious rhetoric but includes elements of radical identity groups and is reactionary.
One of the more rigorous attempts at classification of this modern terrorist is that of Crenshaw \citeyear{crenshaw1981}, who divided them into revolutionaries, nationalists, minority separatists, reformists, anarchists or millenarians, and reactionaries.
She and other authors have heavily emphasized the religious terrorist, with obvious connections to Rapoport’s fourth wave \cite{crenshaw2000}.

To fully understand today's terrorism we have to step back into the 1990s.
The Cold War ended and the last vestiges of hate groups like the KKK were dying out, in their place came anti-government militants.
The Unabomber, Ted Kaczynski, was active during this time, killing 3 and injuring 23 others in a campaign of mail bombing, forcing the media to publish his anti-technology manifesto before being captured in 1995.
1992 and 1993 saw sieges at Ruby Ridge and Waco by government forces, the former ended in 3 deaths, the latter in 76 when the compound of the Branch Davidians, followers of Messiah-claimant David Koresh caught fire.
Two years to the day in response to these sieges, Timothy McVeigh and Terry Nichols, connected to the militia movement and inspired by literature like \emph{The Turner Diaries}, bombed a federal building in Oklahoma City, killing 168 \cite{ling2021}.

These events shifted government and scholarly attention to the danger of domestic terrorism, but Oklahoma City was an outlier in scale. 
Their biggest impact was the quiet influence they had on the nascent conspiracy theory community, which until then had been preoccupied with aliens and JFK.
Alex Jones was inspired by government overreach in these attacks, and began hosting public access television, then radio shows.
William Cooper, precursor to Jones and author of \emph{Behold a Pale Horse}, a best-selling mishmash of conspiracy theories, called into his show (Nichols, McVeigh, and Jones had all been listeners of Cooper's show) \cite{ling2021}. 

Jones spent the late 1990s obsessed with New World Order theories like the Bilderburg Group and Bohemian Grove, then began his rise into the mainstream with the September 11th attacks, which he labelled a false flag that same day, making him one of the first truthers \cite{ling2021}.
His false flag claims continued with Sandy Hook.
By the 2016 presidential election Roger Stone was a regular guest on his show, this culminated with the appearance of Donald Trump in December 2015, Trump would later cite Jones' InfoWars in tweets \cite{finnegan2016}.

Trump's presidency saw the rise of QAnon, which wrapped up a variety of earlier beliefs like Pizzagate and Satan worshipping cabals of politicians.
Q's first post, on October 28, 2017 on the mostly anonymous and infamous imageboard 4chan, claimed the arrest of Hillary Clinton was imminent.
Despite being a false prophet, Q (with a claimed Q-level security clearance) would continue to amass a following via his regular ``drops'' on 4chan and then the more obscure 8chan.
These drops consisted of a variety of vague claims and mottos like ``The Calm Before the Storm'' and ``Where we go one, we go all'' using a general strategy of cold reading which fit into the populist narratives harnessed by Trump and his team \cite{amarasingam2020}.

Q's drops soon spawned a cottage industry of small outlets like the YouTuber ``Praying Medic'' interpreting them.
Trump on multiple occasions acknowledged and supported Q, as have multiple Republican politicians.
General Michael Flynn, former National Security Advisor, implicated and charged due to the Mueller investigation, has recorded himself giving an oath using Q's language \cite{gilbert2021}.
Supporters and symbolism were involved in the January 2021 storming of the US Capitol and the movement is now a global phenomenon.
Various outlets have claimed that Jim Watkins, owner of 8chan (now 8kun) and his son Ron Watkins either are or know Q's identity, the latter ran for Congress in 2022 \cite{schager2021}.

Of note is 8chan's centrality to multiple mass shootings.
Brenton Tarrant posted his manifesto, \emph{Great Replacement}, on the site before live streaming on Facebook an attack on a mosque in Christchurch, New Zealand in March 2019 which killed 51 people (great replacement is itself a racist conspiracy theory).
John Earnest similarly posted his justification on 8chan, identifying it as a radicalizing influence, before killing one and injuring 3 on an attack on a synagogue in Poway, California a month after Tarrant.
Patrick Crusius posted his \emph{The Inconvenient Truth} on the site before killing 23 in a mass shooting at an El Paso Walmart later that summer \cite{baele2023}.
Most recently, Payton Gendron posted his manifesto on Google Docs before murdering 10 in a mass shooting in a Buffalo grocery store in 2022.
He cited great replacement theory, 4chan as a radicalizing influence, and streamed his attack on Twitch \cite{collins2022}.

These attacks speak to a wider issue of stochastic (or random) terrorism.
Terrorists no longer have to be trained by organized groups or work in covert cells, instead they become influenced and supported by extremists online.
The majority of these influencers are just that, influencers, with no intent to take action, but a small minority perceive their circumstances as grim enough to destroy their lives and others.
Reddit in particular has pockets of loathing and hate like incel subreddits which encourage young, disaffected men to see their situation as hopeless and various groups as responsible, an Other worthy of attack.
Furthermore, they achieve a kind of fame in the dissemination of their diatribes and videos, representing the same ``fatal cocktail'' which motivates Islamic suicide bombers: a lack of hope and the promise of a warped glory \cite{victor2003}.

Feeding this are widespread forms of millenarianism, a belief in a coming mass transformation of society.
In the past this was generally the domain of religious movements, messianic claims preceding the time of Christ.
Evangelical Christianity has a deep tradition of this, spawning a number of offshoots in the 1840s including Mormonism with Joseph Smith's prophecies of the destruction of the US government, and the Millerites, whom maintained followers despite repeated prophetic dates that came and went without fulfillment.
Trump and Q latched onto this sentiment with their claims of a coming storm, a Manichaean fight of good versus evil.

Belief of this type is not to restricted to Christianity, Shia Twelver Islam, the state religion of Iran, is a major geopolitical motivator.
Nuclear war, which hung over of generations during the Cold War has seen a return with threats by North Korea and more recently by Russia throughout the war in Ukraine.
Similarly, the fear of climate change has created a cynical ``doomerism'' apparent throughout Reddit, further exacerbated by claims of either widespread job loss due to artificial intelligence, or fears of its dominance of humanity by figures like Elon Musk.
It is this general social anxiety in all its forms in an era of mass social change which fuels random stochastic violence, especially when encouraged by outside actors.

We end this section with a note that while conspiracy theories are having more and more impact, they have been largely neglected by the academic community.
In reference to them, Strukov (2014) via Yablokov: ``with little academic analysis'' \cite{yablokov2015}.
Nearly thirty years earlier, Graumann \citeyear[p. 245]{graumann1987}: ``this topic of intrinsic psychological interest that has been left to history and to other social sciences''.  
Goertzel \citeyear{goertzel1994} states, ``It is puzzling that conspiratorial thinking has been overlooked in the extensive research on authoritarianism which has dominated quantitative work in political psychology since the 1950s''.
Finally, Douglas \citeyear{douglas2019}, in perhaps the most complete cross-discipline literature review, mentions the recency of much research, ``only developed in the last decade''.
Conspiracy theories may have been seen as fringe beliefs before the 1990s, but today they offer some of the most compelling and relevant social science research.
