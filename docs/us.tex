\section{Introduction}
The previous chapter provided a foreign, international case study of pro-Russian superspreaders and their influence on discussion over the Crimean invasion and following conflict.
The Mueller Report, discussed shortly, gives details of a sustained  computational propaganda campaign that engaged with and supported both progressive and conservative causes during the 2016 presidential election.
Besides this foreign interference, the presidential election draws the most coverage and interest of all elections in the United States.
Furthermore, the United States has a history of using new communication technology in campaigns and an active press.
Given that we should expect more partisan activity during presidential elections, what did progressives and conservatives discuss during the 2016 election?
Do the narratives in these discussions match reality or is are they distorted and biased?

Social media is only the leading edge of communication technology, one largely developed within the United States.
The United States, as an open, capitalist democracy has had an outsized impact on the development and commercialization of communication technology during the last two centuries, political adoption has quickly followed.
In this chapter we take a wide scope, hoping to capture the very gradual global development of these tools throughout history.
From the printing press to Twitter, communication has made the world of today feel much smaller than it has been historically and our benefit of hindsight presents a simple narrative of progress which in reality was the emergent result of many interacting cultures over time, an abstract social network.
Correspondingly, the following describes how the political systems of the United States and Russia differ due to an accumulation of historic factors, these old events giving narratives their power.

\subsection{The Mueller Report: Conspiracy, Coordination, Collusion?} 

On March 22, 2019, The Report On The Investigation Into Russian Interference In The 2016 Presidential Election or ``Mueller Report'' was delivered to US Attorney General William Barr \cite{nyt-2019}.
Two days later, Barr sent a four-page summary (the Barr letter) to Congress with the Justice Department's conclusion that there was not sufficient evidence of coordination with Russian operatives during the 2016 election or obstruction of justice by Trump and his associates to warrant further action \cite{barr-2019}.
Mueller criticized Barr's summary as failing to ``capture the nature, context, and substance'' of the 448-page report, leading to ``public confusion'' \cite{wp-2019}.
The report was released with redactions to the public on April 18, 2019 \cite{cnbc-2019}.

Robert Mueller was appointed Special Counsel after Trump dismissed Director of the Federal Bureau of Investigation, James Comey, while Comey and the FBI were investigating Trump and his associates for connections with Russian officials.
Special counsels are appointed when there is a conflict of interest, the most prominent example is Watergate.
Mueller was given authority to investigate connections between Trump associates and Russia, whether Trump obstructed justice with Comey's dismissal, and whether there was Russian interference in the 2016 election \cite[p.11]{mueller}.

Volume 1 of the report details how the Internet Research Agency (IRA), funded by Yuri Prigozhin, in 2014 began a social media campaign of ``information warfare'' and ``active measures'' that were ``designed to provoke and amplify political and social discord in the United States''.
That year IRA employees made trips to the US and obtained information and material used in a campaign aimed at undermining the electoral system.
By 2016, their efforts shifted focus to supporting Sanders and Trump while attacking Clinton; party infighting and a Trump presidency being favorable to Russian interests \cite[p. 14]{mueller}.

Their tactics involved ``specialists'' representing themselves as American individuals and organizations across the political spectrum on various social media websites \cite[p.22]{mueller}.
During 2016, the IRA purchased more than 3,500 ads on Facebook worth \$100,000 to promote causes supported by botnets across platforms: ``United Muslims of America'' had over 300,000 followers, ``{Don't Shoot Us}'' over 250,000, ``Secured Borders'' over 130,000, and, on Twitter, @Pamela\_Moore13 over 70,000. 
US media outlets quoted IRA accounts; Donald Trump Jr., Eric Trump, and Kellyanne Conway cited or retweeted @TEN\_GOP; Roger Stone, Sean Hannity, and Michael Flynn interacted with the same account; Donald Trump's personal account responded to @10\_gop, an alias of the deactivated @TEN\_GOP. 
As of January 2018, Twitter had disclosed 3,814 IRA accounts posting 175,993 tweets in the 10 weeks leading up the the 2016 election; by August 2017 IRA Facebook accounts made over 80,000 posts reaching at least 29 million ``U.S persons'' and up to 126 million people \cite[pp. 26-28]{mueller}.

Fake personas organized dozens of rallies in the US, some with hundreds of attendees throughout 2016, but failed to get official support from the Trump campaign, despite multiple attempts.
These measures were combined with the hacking of members of the Clinton campaign and the Democratic National Convention by Russian intelligence (GRU) using methods like spearphishing - an email disguised as password reset giving unintended access \cite[p.29-37]{mueller}.
This and Clinton-related material was leaked during the campaign by the personas of DCLeaks and Guccifer 2.0, both tied to GRU \cite[p.41]{mueller}.
These personas had repeated contact with Julian Assange; WikiLeaks would twice release this material within hours of the revelation of Trump scandals, and would continue to suggest that the source had been Seth Rich, murder victim, despite knowing otherwise.
GRU similarly gained access to and installed malware on networks of individuals and organizations involved in local and state elections, voting-technology companies, and voter databases \cite[p.46-51]{mueller}.

The report summarizes, ``the Russian government interfered in the 2016 presidential election in sweeping and systematic fashion''.
While political rhetoric is often intentionally obtuse, legal rhetoric by nature should be precise.
The jargon of the report lends itself especially to misrepresentation, using legal terms which have different meanings in a popular, public context.
Collusion is not a ``term of art'' in federal criminal law, but multiple parties involved with the social media campaigns and hacking have been charged with and convicted of conspiracy and other felonies. 
There was no conclusive evidence that the Trump campaign coordinated, or had an agreement with Russian officials, but there was extensive communication and apparent attempts to hide this interaction \cite[p.2-9]{mueller}.
In the case of the IRA social media contacts, there were no charges due to a lack of evidence that anyone that coordinated or communicated with them were aware they were fake US personas operated by Russian specialists \cite[p.35]{mueller}.
However, the Trump campaign welcomed the illegal foreign election interference they were aware of and former National Security Advisor Michael Flynn and Trump campaign chairman Paul Manafort pled guilty to lying about their interactions with Russians \cite[p.9-10]{mueller}.
The report makes no judgement on competence, but it is clear that officials and citizens across the political spectrum were duped throughout the 2016 election cycle using domestic corporations and technology.

\subsection{History}

To understand American political development up to 2016, it is useful to go further back into Russian history.
The states known today as Russia, Belarus, and Ukraine began with the Kievan Rus' polities founded in the 9th century.
The earliest historical source we have of these events is the Primary Chronicle, part history, part legend, written more than a century later of ``bygone years'', not unlike other national epics like the Greek Iliad, Hebrew Bible, or Persian Shahnameh.
Scandinavian Norsemen, known as Varangians, traded with, settled among, and raided the Baltic and Eastern Slavic tribes that lived along the rivers between the Baltic and Black seas.
In the Chronicle's account, the Varangians were driven out by locals whom then fought amongst themselves, leading to their request of the Scandinavians to return under Rurik and his brothers as rulers.
He supposedly took power in Novgorod; certainly and gradually the Varangians Slavicized and adopted local customs, creating a unique ethnicity and culture, the Rus'.
Rurik, said to have died in 879, may not have existed, but the Rurikids remained the ruling dynasty of Kievan Rus' until 1598 \cite[p.1-4,53]{bushkovitch2011}.

The Rus' traded as far east as Baghdad and raided as far south and east as Persian and Turkic settlements near the Caspian Sea; in 860, they attacked Constantinople. 
This raid was led by the founders of Kiev, according to the Chronicle. 
Due to geography the Byzantine Greeks and Rus' made natural trading partners and competitors; Constantinople would be attacked on at least two other occasions over the next century, but treaties were signed and relations normalized \cite[p.41-52]{gleason2009}.
The eastern Church sent missionaries among the Slavs, and Igor of Kiev's wife, Olga, converted and served as regent after his death.
Her son, Sviatoslav, in 965 defeated the Khazars, a unique Jewish Turkic state and basis for modern anti-Semitic theories.
His son, Vladimir the Great, converted to Christianity in 988 and married the Byzantine Emperor Basil's daughter, Anna, cementing the eastern European ties of the Rus' \cite[p.6-8]{bushkovitch2011}.
That year Vladimir sent 6,000 men to help Basil quell a revolt; Basil, not trusting the loyalty of his own men, formally established this Varangian Guard as the emperor's personal bodyguard \cite{gleason2009}.

In this same era, the Norsemen that headed west became known as Vikings.
Repeated raids up French rivers and on Paris led to a peace treaty in 911 in which the Viking Rollo became Duke of Normandy, a vassal of the King of France, and was baptized.
His descendant William conquered England in 1066 and this Norman elite eventually blended with existing waves of Anglo-Saxon, Viking, Roman, and Celtic inhabitants to form a unique ethnicity and culture, the English \cite{rowley2022}.
It was from Western Europe that the United States took its political culture and the WASP (White Anglo-Saxon Protestant) core until recently acted as owners and custodians of political power, now challenged (and evolving) due to new media.

Norse settlement extended as far west as Greenland and Newfoundland in the 11th century.
Normans ruled portions of Scotland and Ireland, captured Sicily and Malta, ended Byzantine rule in southern Italy, established states in Antioch in Syria and on Cyprus during the Crusades, and took part in the Reconquista in Spain \cite{rowley2022}.
Descendants of Scandinavians had similar cultural origins but over time the orientation of the former Vikings to the west and the once Varangians to the east created strong distinctions.
The split between the western Catholic church and the eastern Orthodox church in 1054 along not only confessional but also political, linguistic, and national lines is an early example of this European divide obvious even today \cite[p.8-9]{bushkovitch2011}.

This more general process of a nomadic and military elite being assimilated by a native population into new synthetic cultures is common.
The Bulgars similarly Slavicized, the Magyars (Hungarians) kept their language, Anatolia and large parts of Central Asia Turkified, Turkic tribes and the Mongols of the Ilkhanate Persianized, and the Mongols of Kublai Khan established the modern capital at Beijing, in time becoming China's Yuan dynasty \cite{golden1992}.

The Rus' states under the Rurikid princes were also defeated by the Mongols. Kiev was razed in 1240, and for hundreds of years the Golden Horde dominated while the Rus' were devastated by their armies and heavy taxation.
During this time Kiev, Novgorod, and other cities lost power, Moscow rose to prominence in the northeast, and the Rus' began the divide into their modern nationalities \cite[p.19-23]{bushkovitch2011}.

In 1533, when Ivan IV (the Terrible) became Grand Prince of Muscovy, his territory was an isolated and unknown country between the Dnieper and Volga rivers.
He conquered khanates to the east, was crowned tsar in 1547, and expanded Russia to the Ural Mountains \cite[p.47-50]{bushkovitch2011}.
To the south, from the plains of the Eurasian steppe, Tatars made regular raids deep into Russian territory along the Muravsky, Izyumsky, and other ``trails'', and in 1571 Moscow was burned with tens of thousands of Russian victims and prisoners \cite[p.264-277]{madariaga2006}.
Raids continued into the 18th century and were not completely ended until the annexation of the the Wild Fields of Ukraine and Crimea during the reign of Catherine the Great. This area eventually became ``Novorussia'' through an influx of settlers and construction; the Donbas (Donets Basin) became an industrial center in the late 19th century \cite[p.109,122]{bushkovitch2011} \cite{sunderland2004}.

Despite or because of similar threats, the tsardom and then empire began a slow, continuous expansion to the west further into Europe, east towards the Pacific Ocean, and south to the Caucasus Mountains, Central Asia, and Black Sea.
Yermak Timoveyevich led the first campaign into Siberia in 1582; after 1613, Russia grew at a rate of 20,000 square miles per year \cite[p.53]{bushkovitch2011} \cite{montefiore2017}.
Russian settlement reached its furthest extent on the Pacific at Fort Ross in 1814, 90 miles from today's San Francisco.
The Russian-American Company operating the fort sold it to John Sutter in 1841, seven years later gold was found at another of his properties beginning the California Gold Rush \cite{parkman1996}.

This Russian expansion mirrors that of the United States from isolated settlements to an empire touching the Pacific, overwhelming people groups on either side of the Bering Strait.
However, their political institutions developed very differently, even though
Russians were aware of democratic forms of governance.
Novgorod was a republic with a popular assembly that elected a prince from among the boyars \cite[p.12]{bushkovitch2011}.
Peter the Great opened the country to western Europe, founding Saint Petersburg in 1703 on the Gulf of Finland, and the empire consumed states with democratic traditions  such as the the Cossacks of Ukraine and Polish-Lithuanianian Commonwealth in the 18th century.
Despite these influences, it remained a monarchical state dominated by a limited nobility, even more authoritarian than most other European monarchies of the time \cite[p.28, 65]{bushkovitch2011}.
The Emperor adopted the term autocrat from the Byzantine Greeks, emphasizing the absolute power of the sovereign to lead and protect the Orthodox Church and its followers, Russian people, and Slavic nations \cite[p. 39]{bushkovitch2011}.
This responsibility and justification was used throughout the 19th and 20th centuries, leading to the Crimean War (1853-1856) and repeated conflict in the Balkans and southeastern Europe \cite{montefiore2017}.

When Napoleon's invasion of Russia failed in 1812, Russian forces captured Paris and at the Congress of Vienna in 1815, the Russian Empire became a recognized great power that kept stability through the Concert of Europe and the Holy Alliance.
In this role, the Emperor Nicholas became known as the gendarme (policeman) of the continent.
His first act was to end an attempted coup by reformist officers known as the Decembrists in 1825, he put down a rebellion and abolished the constitution of Poland in 1830, then Russian forces were requested to crush a liberal revolt in Hungary in 1849 \cite[p. 149-166]{bushkovitch2011}.
Lincoln captured liberal sentiment in 1855 while writing of slavery and the anti-immigrant Know-Nothing party: ``When it comes to this I should prefer emigrating to some country where they make no pretense of loving liberty – to Russia, for instance, where despotism can be taken pure, and without the base alloy of hypocrisy.'' \cite[vol. 2, p. 323]{lincoln2008}

Yet Russia was slowly modernizing and liberalizing; in 1861, Emperor Alexander II freed the serfs from being tied to the land, ending an outdated, ancient feudal system which had limited industrialization \cite[p. 189-192]{bushkovitch2011}.
In the United States, the first transcontinental railroad was finished in 1869, while the Russian Trans-Siberian railway opened in 1904, connecting Europe to the Pacific \cite{bowman1957}.
Due to its large population and material resources, the country always had great wealth, but mass industrialization changed the economy and society as the coal, iron, steel, textile, and petroleum industries became some of the largest in the world (the fortune that funded the Nobel prize came from the oil fields of Baku) \cite[pp. 837-844]{portal1965}.
Saint Petersburg would double in population between 1890 and 1914 as people flocked to the cities to join a growing working class; nearly 30\% of factory workers were women \cite[p. 208-227]{bushkovitch2011}.
Defeat in the Russo-Japanese War in 1905 forced Nicholas II to establish a representative legislature, the Duma \cite[p.269,211]{bushkovitch2011}.
Though it had only been ruled by Rurikids and Romanovs for more than a thousand years, by the start of World War 1, few countries had faster economic growth or more political potential than Russia.

\subsection{The Federalist Papers}

Political evolution in the United States was to a much greater extent by intentional design.
The founding generation Lincoln called the ``patriots of seventy-six'' had this Russian and other political histories as a context and guide when putting together the Constitution seventy years earlier \cite[vol. 1, p. 112]{lincoln2008}.
Some of the best material we have on the thinking behind its design can be found in The Federalist Papers.
Part descriptive, part persuasive, the papers were in total 85 essays written by Alexander Hamilton, James Madison, and John Jay as ``Publius'' published between October 1787 and the summer of 1788 in New York newspapers in support of the ratification of the Constitution \cite[p. 2]{mosteller2012}.
Though not a complete picture of popular political views at the time, this outcome of the convention of 1787 was ratified by the states the next year.
The primary arguments against the papers came from a number of essays, several by ``Brutus'' and ``Cato'', which would years later be collected as ``The Anti-Federalist Papers''.
Despite losing the overall argument, their main contribution was a stand for a Bill of Rights; George Clinton, vice president under Jefferson and Madison, was one notable writer \cite{borden1965}.

The Federalist Papers make clear the necessity of revising the existing confederacy of states established after the revolution, in particular strengthening the federal government by giving it the power to raise taxes and army, and to act as a separate but not dominant power over the state governments.
Shay's Rebellion of 1787, led by a Revolutionary War veteran over high taxes and nonpayment was only put down after a private army was raised.
Federalist 1 poses this event as proof of the failure of the existing government and presents two options: either the Constitution would be ratified or inevitably the states would form separate regional confederacies \cite[n. 1]{fed}.
Washington, who admired and had followed the example of the Roman dictator Cincinnatus returning to work on his farm, was thought to be considering a third option, constitutional monarchy \cite[ch. 3]{feldman2017}.

Washington's motives are understandable though the lens of papers.
They are a model of how the founders perceived the world as well as their audience, those eligible to vote.
Following the Greek and Roman traditions of criticizing Athenian democracy, the distinction is made between the ``true principles of republican government'' and democracies \cite[n. 1]{fed}.
Democracies are defined as states where individuals directly vote on issues and are limited to small geographical areas (today's ``direct democracy'');
republics are states where voters elect representatives and smaller states are bound together into a greater whole.
The term ``confederate republic'' is borrowed from the political theorist, Montesquieu \cite[n. 9]{fed}.

The democracies of Greece and the maritime states of Italy are described as having been ``in a state of perpetual vibration, between the extremes of tyranny and anarchy'' due to infighting \cite[n. 9]{fed}.
Within small states it takes small amounts of influence and power to gain control, by increasing the pool of representatives and size of the state, ``true'' republics improve the quality of the people governing, disperse and redirect factional conflicts, and ``refine'' public views \cite[n. 2, 10]{fed}.
The smaller states within the larger republic receive the advantages of internal standardization and limit petty intranational conflicts and, more importantly, present a united front to outside powers, military and commercial.
Echoing an understanding of the necessity of the sovereign via Hobbes and Locke, this republic is said to have the external advantages of a monarchy \cite[n. 9]{fed}.

The founders were familiar with analyses of government going at least as far back as Plato's \emph{Republic} in which pure democracy, or the inevitable rule by sectarian demagogues, was only preferable to tyranny \cite{plato}.
Cicero's \emph{De re publica}, modeled after the former work, praised the late Roman republic's mixed government \cite{cicero}.
Polybius, a Greek historian writing in the 2nd century BCE, similarly lauds Rome's mixed constitution (the Senate and the people) and separation of powers \cite{polybius}.
Of this knowledge ``imperfectly known by the ancients'', the papers consider distribution of power into distinct departments, legislative checks and balances, courts with accountable judges, and representative legislatures as either ``wholly new discoveries'' or primarily modern advances.
They make no claims about the value of a written constitution but note that ``the science of politics'' had made much progress since the times of the Romans \cite[n. 9]{fed}.

The confederate republic is an advantage in foreign affairs by elevating the stature of the state, the operation of government, and those chosen to operate it.
``Safety from external danger is the most powerful direction of national conduct'', however, considering the country's ``insulated'' geographic position, external danger would most likely result from internal instability caused by faction.
States in constant conflict require standing armies for national defense, the ``military state becomes elevated above the civil'', and citizens view the military as superior.
The greatest danger of this happening is not from foreign powers, but due to the splintering of the union; the resulting smaller confederacies European powers would play against each other \cite[n. 8]{fed}.

To defend against this, the Constitution was designed with multiple failsafes including seperation of powers, checks and balances, and overlapping forms of representation in the houses of Congress.
The geographic size of the republic is another, the hardest to overcome.
If a section of the union experiences abuses of power, popular insurrection, or a ``religious sect degenerate into a political party'', other states can reform or aid and bring together the greatest resources to do so \cite[n. 10]{fed}.
Just as importantly, the Constitution offers a guarantee of the states themselves ``against changes to be effected by violence'' \cite[n. 21]{fed}.

These safeguards are necessary due to human behavior within which are the ``latent causes of faction...the most frivolous and fanciful'' issues leading to conflict \cite[n. 10]{fed}.
This is because ``men are ambitious, vindictive, and rapacious'', their passions ``will not conform to the dictates of reason and justice'', ``neither moral nor religious motives'' are a restraint, nations will go to war both whenever there is something to gain, and when there is not \cite[n. 6, 15, 4]{fed}.
The ``love of power'', ``self-love'', and pride of men and nations which ``naturally disposes them to justify all their actions'' are but some of the challenges of human interaction \cite[n. 10, 3]{fed}.
This weakness includes the fallibility of our reason, a driver of faction though differences in human opinion, distrust causing distrust \cite[n. 10]{fed}.
Social media similarly is designed to harness human behavior, but the incentives are much different.
Platforms benefit from virality and conflict, individuals benefit from  outrageous claims and justification of their acts, as such open democratic societies are more vulnerable to mass discussion than more constrained authoritarian countries.

The writers of the papers were are aware of other weaknesses of a republic, specifically vulnerability to foreign corruption.
It is pointed out that there have been as many popular wars as royal, commercial republics are just as quick to go to war for gain, commerce only changing the objectives of war \cite[n. 6]{fed}.
In peaceful and warlike republics alike good leaders will not always be available.
The ``class of men in every State'' that resists change is noted, the importance of moderation emphasized, and the awareness that their own discussions of government were tainted by ``considerations not connected with the public good''.
The papers stress that the Constitutional Convention was conducted through extensive and ``sedate and candid'' consideration during a time of peace, but even in this ideal situation, it was an experiment, and it was no surprise that the experiment that was the current government was flawed \cite[n. 1]{fed}.
In summary, the Constitution is designed to engage human behavior, not ignore it. 
Both party and faction, painted as negatives throughout, are crucial to the regulation of different interests, ``the principal task of modern legislation'' \cite[n. 10]{fed}.
To do this the state and federal governments must engage human emotions, the ``hopes and fears of individuals'' \cite[n. 16]{fed}.

Extensive historical examples are provided to arrive at these ``solid conclusions'', experience being the ``oracle of truth'' \cite[n. 8, 20]{fed}.
Xerxes, Plutarch, the many ancient Greek confederacies are discussed; Sparta, Athens, Rome, and Carthage categorized as republics; Rome ``never sated'' by war \cite[n. 18, 6]{fed}.
Pericles is singled out as a poor leader, Cardinal Woolsey, prime minister of Henry VIII, an ``instrument'' and ``dupe'' of Emperor Charles V of the Holy Roman Empire \cite[n. 6]{fed}.
The structure and history of that empire dating back to Charlemagne, its diets and circles as well as dysfunctions leading to the Thirty Years' War and eventual Peace of Westphalia (1648) elaborated on at length \cite[n. 19]{fed}.
The authors enumerate several bad female rulers and criticize Venice, Genoa and other democracies: the Swiss cantons, ``scarcely...a confederacy'', Poland, the United Netherlands and its stadtsholdership systematic failures \cite[n. 19, 20]{fed}.
The evolution of Great Britain is used to give context, an opportunity for the United States to ``profit by their experience'' \cite[n. 5]{fed}.
Strongest criticism is reserved for ``idle theories'' and misled ``profound philosophers''; the proposed Constitution is a practical document designed using the knowledge of the era \cite[n. 6, 11]{fed}.
We again contrast this very well-informed and historically aware approach with social media networks, which in many ways are designed for capture.
Reddit, Facebook, Twitter, and others have only the most basic of voting systems or governance but exist parallel to other real-world networks of voters, parties, and governments which they have much influence over.

\subsection{Communication is Key}

The Constitution was not designed with social media in mind with its immediate, global reach.
The key behind the enlargement of the republic was the difficulty of communication, it was this time and space which prevented factions from spreading.
The papers concede that the minority can influence national discourse and introduce ``tedious delays'' or ``contemptible compromises of the public good''; ``good may be prevented'' by constitutional restrictions requiring agreement of large numbers, but the brake is the point \cite[n. 22]{fed}.
The Constitution does not ignore passions but ``by number and local situation'' dangerous small factions are ``unable to concert and carry into effect schemes of oppression''. 
Schemes across large areas are difficult, distrust increases in proportion to the number of people involved and requiring agreement; as even local assemblies can not agree on important issues, the odds of any sizable or diverse group staying a coherent force over time is low, especially over distance \cite[n. 10]{fed}.
This locality argument is applied to state governments, too: local governments have the most interaction with the people, this familiarity presumably leading to a natural bias for one's own state.
When the people of a state favor federal action over state, there is a signal the state is in error \cite[n. 17, p. 109]{fed}.
The idea is not to abolish local interests but expects they will stay local and cannot capture the government; the Constitution is an attempt to harness often unproductive but natural human behavior as an intentional, productive force. 

Of course, communication and information technology change over time.
We can observe some of these innovations and their effects on societies and politics with our own sampling of history up to the United States in 2016.
Writing coincided with the rise of civilizations throughout the world: Sumer in Mesopotamia, Egypt, northeastern China, and southern Mexico and Guatemala independently developed writing, and proto-writing systems existed globally \cite{fischer2003}.
Writing enabled cultural, economic, and political growth and for societies to organize at previously impossible levels. 
Until very recently, the fastest way to transport people or large amounts of writing, text, and information were the same: ship or horse.
The ancient Persians and Mongols were famous for their standardized circuits of relays where a message could move from one fresh horse and rider to another across thousands of miles \cite[8.98]{herodotus} \cite{shim2014}.
This speed of communication enabled mobilization against unorganized states on their borders and also efficiently spread rumor or imperial propaganda; the tsars continued this postal (yam) network \cite{smith1970}.
The sultans of Cairo received daily messages through a series of pigeon stations, and though they lacked horses, the Inca Empire used a network of runners, the chasquis, trained to use a device, the quipu, to encode and decode messages \cite[p. 60]{bloom2001} \cite[pp. 14-15]{fischer2003}.

Though not as information-dense as writing, smoke and other optical signaling has been used much longer.
The aforementioned Polybius further developed an earlier Greek square or ``checkerboard'', a grid of letters and digits which could produce open-ended messages with fire signals \cite[10.43-45]{feldman2017}.
We know of Polybius and his work thanks to the Hellenization of the ancient world in the wake of Alexander's conquest of the Persian Empire.
Greek became a dominant language of culture, philosophy, science, and literature and would be preserved and develop under various states and religions into the Renaissance.

Similarly, Arabic, specifically the high-tongue of the poets of the interior desert, enabled the people of the Arabian peninsula to consolidate a vast empire.
While originally an oral culture where writing was an aid to memory,
the definitive Arabic text, the Quran, made them ``legible'', Arabization spread common culture, religion, and technology from Spain to India \cite[pp. XVIII-8]{mackintosh2019}.
The language had such success serving as a conduit and medium of ideas and knowledge, whether Greek or Indian, that it is easy to overlook that many of the most famous ``Arabic'' scholars were in fact culturally Turkish and Persian, using it as a second tongue \cite{starr2015}. 
The polymath al-Khwarizmi is but one example; called the father of algebra, his book on Indian numerals and decimal points introduced those concepts to Europe, and the term algorithm is derived from his name \cite{arndt1983} \cite[p. 12]{bloom2001}.

One massive advance, paper, was adopted by Islamic states after encountering this originally Chinese technology in Central Asia; the oldest extant paper Quran dates from about 950 \cite[p. 106]{bloom2001}.
It had several advantages compared to the parchment and vellum still in use in Europe and the papyrus used elsewhere.
Paper was much cheaper, could be produced anywhere, did not fray on the edges and was foldable unlike papyrus, and because ink soaked into it, was harder to forge documents on than alternatives. 
The Abbasid caliphate in Baghdad built the first paper mill in the city by 795 in order to meet the needs of a growing bureaucracy \cite[pp. 48-69]{bloom2001}.
This spawned a market for paper and books, paper sizes became standardized ( al-Baghdadi and half-Baghdadi), local varieties were preferred (Persian and Chinese paper were very high quality), and new genres (cookbooks) and masterpieces (One Thousand and One Nights) were created \cite[pp. 48-69]{bloom2001}.
The Islamic world had numerous private and public libraries: al-Khwarizmi became head of the House of Wisdom in 820 in Baghdad, al-Hakam II of Spain was said to have 400,000 books in the 950s, al-Afdal, vizier of the Fatimids in Cairo had over 500,000 books in his library when he died in 1121, and when Saladin conquered that dynasty 50 years later, he was said to have found over 1.6 million books.
In contrast, the richest library in Europe, at Sorbonne, had 338 books and another 1,728 works in 1338 \cite[pp. 117-22]{bloom2001}.
These innovations we take for granted today played a major role in progress and dominance of Islamic science for centuries \cite{hunter1978}.

Indeed, even formats like the codex and then book were gradual inventions, improvements on the ancient scroll.
Eventually Europe adopted paper making through Italy and Spain; the ``ream'' of paper made its way to English by Old French (rayme) via Spanish (risma) via Arabic (rizma) \cite[p. 9]{bloom2001}.
There the paper industry would continue to develop along with modern notions of commerce and banking, and the Italians may have sold paper at a loss to capture the Egyptian market \cite[pp. 205-212]{bloom2001}.
In Iraq, the Baghdadi paper industry rebounded after its destruction by the Mongols in 1258, but ended with Timur's sack of the city in 1401 \cite[pp. 53-56]{bloom2001}.
Improvements in communication technology after this time were connected to the wider changes of the Renaissance in Western Europe.

It was this culmination of different events by which Johannes Gutenberg invented the printing press circa 1440, a combination of his creations including movable type and an oil-based ink with existing technology, like the press.
In 1455 the run of under 200 copies of the ``Gutenberg Bible'' had been printed in Mainz, Germany \cite[p. 15]{bloom2001} \cite{chaplin2005}.
By 1500, there were up to 1,000 printing presses in Western Europe and up to 8 million printed books \cite[pp. 13-17]{eisenstein2005}.
Martin Luther's 95 Theses of 1517 started the Reformation and with it an outpouring of literature that created and exacerbated divides in Germany, the resulting Thirty Years' War, and the Westphalian peace which created today's world of sovereign states \cite[ch. 6]{eisenstein2005}.

Other states avoided upheaval through slow adoption: in 1553, Ivan Fyodorov established Russia's first print house, the only official press until the time of Peter the Great \cite[pp. 95-97]{appel1987}.
Even though the cursive Arabic script had been printed, languages that used it (Persian and Turkish) preferred the art of calligraphy; more importantly, there were an estimated 80,000 copyists in Istanbul (formerly Constantinople) in 1682 with incentive and influence to maintain their position.
The first printing press was established in the capital by Jews fleeing Spain in 1492, and the Ottoman Empire allowed presses in the ethnic millets (nations) of Greeks and other minorities in the empire; however, the first state-run press did not open until after 1710. 
When it closed in 1742, the Quran was illegal to print; the first complete Quran for Muslims by Muslims was published in 1787 in St. Petersburg \cite[pp. 218-222, 10]{bloom2001}.

Printing saw continued improvements, reducing costs and increasing output.
Federalist 2 mentions how the press ``began to teem'' with pamphlets and weekly papers in response to the Congress of 1775 \cite{fed}.
In June 1789 alone, there were hundreds of new papers and pamphlets in Paris; these were witness to and drove the revolution in France, only slowing down due to state censorship \cite[ch. 8]{popkin2019}.
During the next century, the Industrial Revolution pushed this innovation faster; bleaching agents, the ability to extract paper fiber from wood, the shift from movable type to lithography, and the development of new presses powered by steam engines occurred by 1840 \cite[pp. 3-5, 224]{bloom2001}.
The various national revolutions throughout Europe, the expansion of the franchise in democracies, the spread of Marxism and anarchist movements, all were powered by cheap, fast printing and a rising middle class.

The 19th century also saw the invention and spread of trains, telegraphy, photography, and the telephone, changing politics, especially in democracies where politicians sought to connect with the people.
The development of railways across Britain in the 1840s and 1850s made the press and London's Fleet Street a 'Fourth Estate', the 'aggregated intelligence of society'.
French and Russian claims over religious sites in Jerusalem, centuries of war between the Ottoman Empire and Russia, and genuine Russophobia came to a head in the 1850s when the press pushed the British into the Crimean War.
Lord Palmerston, the Prime Minister, has been described as the first modern politician in cultivating the press and public \cite[pp. 147-149]{figes2011}.
William Gladstone, four-time PM, as part of championing the common worker took to photography early in its life, and his Midlothian Campaign of 1880 was innovative in using a series of speeches timed for maximum coverage catering to journalist deadlines; it is considered the first modern political campaign \cite[p. 204]{brighton2015}.

In the United States, where presidential candidates had been for so long isolated from campaigning, William McKinley gave speeches during the campaign of 1896 to hundreds of thousands of visitors from his front porch in Canton, Ohio; many came by subsidized train rides \cite[p. 52]{harpine2006} \cite[pp. 130-131]{williams2010}.
During his administration, William Randolph Hearst's papers blamed the sinking of the USS Maine on a Spanish mine, inflaming the coming Spanish-American War.
Hearst and Joseph Pulitzer, in constant competition with other publishers to reach audiences with their daily newspapers, engaged in the scandalous and poorly-sourced content that came to be known as ``yellow'' journalism \cite[pp. 4-5, 219]{thomas2010}.
Hearst used his media empire to support liberal causes in youth as strongly as he supported conservative causes as he aged; his media outlets fueled the public paranoia that led to the rise of Joseph McCarthy \cite{carlisle1996}.

McKinley's successor, Theodore Roosevelt, Jr., was the first president to meet regularly, often daily, with the media \cite{juergens1982}.
Roosevelt and Woodrow Wilson campaigned extensively by train; the latter suffered a stroke during a planned tour of 10,000 miles across 29 cities in 4 weeks pushing for the ratification of the League of Nations by the Senate in the late summer of 1919 \cite[p. 15]{berg2013}.
Franklin Roosevelt, cousin of Theodore and member of the Wilson administration, was the first president to harness the radio, his Fireside Chats reaching constituents in their homes \cite{craig2000}.

His administration also provides an example of innovation not being equally distributed or understood.
During the 1936 presidential campaign, the \emph{Literary Digest} biased their sample by conducting poll by phone, a novelty only some could afford, giving the wrong prediction of a landside victory by Roosevelt's opponent.
This, at least, has been an ``infamous'' popular cautionary tale told in books and articles for years; comparison of this data with a 1937 Gallup poll suggests phone ownership was higher than thought, the survey flawed for other reasons \cite{squire1988}.

The October 7, 1960 presidential debate was the first televised; another oft-cited but unsupported story is that listeners thought that Nixon won the debate while television viewers preferred Kennedy's youth and vitality \cite{vancil1987}.
In 1969, ARPANET, the predecessor to today's Internet went live and connected four universities across the US \cite{denning1989}.
During the 1980s and 1990s, cable television and talk radio catered to partisans in ways network television could not, playing a role in the resurgence of the Republican party under Newt Gingrich \cite{lee2001}.

The Clinton administration used modern media and the new World Wide Web and was able to iterate and react to public opinion through extensive (even daily) polling \cite{heith2004}.
2008 was the first presidential campaign to feature social media sites like Facebook, Twitter, and Reddit.
While Obama and McCain spent extensively on traditional outlets like television advertising, the Obama campaign was lauded for its effective messaging and application of technology to traditional field organizing, mobilization, and outreach \cite{hendricks2010, mckenna2014}.

The last decade has seen the adoption of tech across the political spectrum: in 2010 Scott Brown used a botnet against Martha Coakley in the race for senator in Massachusetts, in 2012 the Romney campaign was accused of buying followers on Twitter, and in 2016 Ted Cruz's campaign had identified a Trump botnet attacking Cruz.
Researchers ``doing politics'' and interacting with campaigns discussed the outfit to be known as Cambridge Analytica with a Cruz campaign employee and interviewed another political operative that admitted these modern techniques were common across campaigns yet skeptical they had any impact \cite[pp. 194-195]{woolley2018}.
Though this impact is difficult to register directly in votes, the 2016 presidential campaign was the natural result of this long process of the democratization of media and society.
Reddit is the leading edge of this democratization and offers the unique ability to capture this change during a limited window of time.
Next, we attempt to measure the activity and influence of progressive and conservative users on the site.